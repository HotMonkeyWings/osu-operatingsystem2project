\documentclass[letterpaper,10pt,draftclsnofoot,titlepage,onecolumn]{IEEEtran}

\usepackage{graphicx}                                        
\usepackage{amssymb}                                         
\usepackage{amsmath}                                         
\usepackage{amsthm}                                          

\usepackage{alltt}                                           
\usepackage{float}
\usepackage{color}
\usepackage{url}

\usepackage{balance}
\usepackage[TABBOTCAP, tight]{subfigure}
\usepackage{enumitem}
\usepackage{pstricks, pst-node}
\usepackage{array}
\usepackage{tabulary}

\usepackage{geometry}
\geometry{textheight=8.5in, textwidth=6in}


\newcommand{\cred}[1]{{\color{red}#1}}
\newcommand{\cblue}[1]{{\color{blue}#1}}

\newcommand{\toc}{\tableofcontents}


\def\name{Group 29}

\parindent = 0.0 in
\parskip = 0.1 in

\begin{document}
	
	\begin{titlepage}
		\centering
		{\scshape\LARGE \textbf{CSS 444 HW 4 - The SLOB SLAB}\par}
		{\LARGE\itshape \textbf{Kyle Collins and Jonathan Harijanto}\par}
		{\LARGE Fall 2016\par}
		{\LARGE \today\par}
		
	\vfill		
		
\begin{abstract}
The purpose of this paper is to describe the design and implementation of the SLOB Best Fit Algorithim for the default yocto Linux kernel as well as the testing methodology used to compare the efficiency of the Best Fit Algorithim to the First-Fit algorithim. This paper will also cover the reasoning for the assignment, lessons learned, and the work/commit log for this assignment. 
\end{abstract}
		
\end{titlepage}
	
\clearpage
\tableofcontents
	
\clearpage
\section{Design Plan}

	Our plan to approach this assignment is as follows. First, we will examine the original First-Fit algorithim as well as the SLOB code in the slob.c file to understand what is occuring and better design our algorithim. We will then create a pseudo-code Best-Fit algorithim and from that implement it into a new slob.c file. From there, we will implement the new slob.c file into our virtual machine. We will then test the new slob.c file to ensure that it is running properly. Finally, we will construct a test file that will be used to compare the space efficieny of the algorithim. We will log the results of the test file and include them in our report and state any noticable differences or observations between the two algorithims. 
 
\clearpage
	\section{Questions}
	
	\textbf{Reasoning For Assignment}\\
	
	The reasoning for this assignment is to familarize the student with memory allocation in Linux, specifically the SLOB SLOB, as well general memory allocation algorithims. Additionally, the testing and comparison of the space efficiency of the two algorithims will provide the student with the opportunity to better understand the impact that poorly constructed memory allocation algorithims can have on a system. 

		
	\textbf{Design and Programming Approach}\\

	As stated previously, the inital approach was to examine the original slob.c file. From there, we determined that the appropiate algorithimic modifications should be made in the *slob\_alloc function. This is the function that actually allocates available memory for use. We then designed a brute force alogorithim to be used to determine the best fit for a given set of data when given a list of possible memory allocations. The pseudocode generates is as follows: \\
	
	bestFit = MAX\_BLOCK\_SIZE
	For(blocks) \\
		if(slob.size - blocks[i].size $<$ bestFit \&\& (slob.size - blocks[i]) $>$ 0)\\
			bestFit = blocks[i] \\
	end \\

	if(bestFit == MAX\_BLOCK\_SIZE) \\
		print "Not enough room"	\\
		return \\
	alloc(slob,bestFit)\\
	
	Essentially, what the pseudocode is trying to convey is a search for the minimum difference between the block to be allocated and the free block space. For instance, if a block to be allocated has a size of 50 bytes and there are free blocks of 75, 100, 65, and 51 bytes, the minimum will be updated to 75, then 65, then 51. After iterating through the free block list the program will allocate the necessary block in the minimum block space. This is a brute force solution so it would take O(n) where n is the number of free blocks in the list.\\
	After turning the above pseduocode into actual code and implementing it into our new slob.c file we loaded yocto Linux with our new SLOB Best-Fit allocator and attempted to run it to test for errors. We then created a test file and usedthis test file to compare the space efficieny of the two algorithims.\\ 
 
	\textbf{Testing Methodology}\\
	
	Our initial testing methodology focused on trouble shooting errors that were created when we tried to load yocto Linux with our new slob.c file. Afterwards, we created a new .c file that can be used to test the space efficieny of the two algorithims. This test file displays the claimed memory, free memory, and fragmentation that has been generated after each iteration of our SLOB algorithims and we will cover the results in the demo. 


	\textbf{Lessons Learned}\\

	This assignment primarly focused on memory allocation in Linux. We learned about the efficieny between the two algorithims, First-Fit and Best-Fit, and how Best-Fit is a better algorithim that can reduce fragmentation of the system. Additionally, examining the slob.c code taught us a great deal about how memory can be physically allocated on the computer and introduced us to different manners of allocating this physical memory. 


	\clearpage

	\section{Version Control Log}
	
\begin{center}

\begin{tabular}{| m{2cm} | m{8cm} | m{4cm} | } 
\hline
 Date & Commit & Message \\ [0.5ex] 
 \hline\hline
2016-11-13 &  &  \\
 \hline 
  2016-11-12 &  & \\
 \hline
  2016-11-11 &  &  \\
 \hline
  2016-11-09 &  &  \\
 \hline
   2016-11-02 &  &  \\
 \hline
 
\end{tabular}

\end{center}
\hfill\\
N.B. We are using: git log --pretty=format:"\%h \%ad\%x09\%an\%x09\%s" --date=short
	
\clearpage

\section{Work Log}
\begin{center}
\begin{tabular}{ |m{2cm}|m{10cm}|m{2cm}| }
\hline
When & What & Duration \\ \hline
11/22/2016
& We met at Kelley computer lab to discuss and identify what needed to be done to complete the project. 
Also, we did some research about SLOB from the internet and Linux Kernel Development book.
At the end of the meeting, both of us understood the basic concept of SLOB.
& 2 hours\\ 
\hline

11/23/2016 
& We decided to take a look at the slob.c file. 
We also made our first commit because we are going to use the original slob.c file as our template.
Lastly, we spent the last one hour to add memory usage system call and change the algorithm in the slob.c file.
& 2 hours\\ 
\hline

11/26/2016 
& We're still not sure with our best-fit algorithm, but we decided to test it anyway. 
Thus, we started to write c file that could test memory fragmentation. 
Furthermore, we modified the kernel configuration so it will use SLOB instead of SLAB.
& 2 hours\\ 
\hline

11/27/2016 
& We added and removed few lines of code in our slob.c because we noticed some logical error
during the execution.
After we are finished with the code, we started to write the document and answer the questions.
& 3 hours\\ 
\hline

\end{tabular}
\end{center}
\end{document}

