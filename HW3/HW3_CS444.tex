\documentclass[letterpaper,10pt,draftclsnofoot,titlepage,onecolumn]{IEEEtran}

\usepackage{graphicx}                                        
\usepackage{amssymb}                                         
\usepackage{amsmath}                                         
\usepackage{amsthm}                                          

\usepackage{alltt}                                           
\usepackage{float}
\usepackage{color}
\usepackage{url}

\usepackage{balance}
\usepackage[TABBOTCAP, tight]{subfigure}
\usepackage{enumitem}
\usepackage{pstricks, pst-node}
\usepackage{array}
\usepackage{tabulary}

\usepackage{geometry}
\geometry{textheight=8.5in, textwidth=6in}


\newcommand{\cred}[1]{{\color{red}#1}}
\newcommand{\cblue}[1]{{\color{blue}#1}}

\newcommand{\toc}{\tableofcontents}


\def\name{Group 29}

\parindent = 0.0 in
\parskip = 0.1 in

\begin{document}
	
	\begin{titlepage}
		\centering
		{\scshape\LARGE \textbf{CSS 444 HW 3 - The Kernel Crypto API}\par}
		{\LARGE\itshape \textbf{Kyle Collins and Jonathan Harijanto}\par}
		{\LARGE Fall 2016\par}
		{\LARGE \today\par}
		
	\vfill		
		
		\begin{abstract}
The purpose of this paper is describe the process of creating an encrypted RAM Disk device driver for an Linux virtual machine, specifically Yocto Linux. 
This paper will cover the design methodology for the driver, implementation of the driver into the virtual machine, testing methodology for the driver, and the knowledge that was gained from completing this project. 
Additionally, this project contains a work log that illustrates time spent on the project and a version control log that documents progress made.
 		\end{abstract}
		
	\end{titlepage}
	
	\clearpage
	\tableofcontents
	
	\clearpage
	\section{Design Plan}
	
We began this project by identifying the appropriate steps we would need to take in order to build the encrypted RAM Disk device driver and implement it into the virtual machine. 
The steps we identified were as follows.
 	 
 	 \begin{enumerate}
 	 	\item Learn about RAM disk device drivers.
 	 	\item Create a new RAM disk device driver module. 
 	 	\item Learn how to incorporate the new module into the virtual machine.
 	 	\item Load the module into the virtual machine.
 		\item Test the new module.
 	 	\item Learn about the Linux Cryptographic API.
 	 	\item Implement the Linux Cryptographic
 	 	\item Final test of the new module.
 	 \end{enumerate}
 
By following these steps we were able to build the new driver and insert it as a module into the virtual machine. 
Further details on the design of the driver and the cryptography associated with it can be found in the Design and Programming Approach section and the tests we conducted can be found in the Testing Methodology section. \\


	\clearpage
	\section{Questions}
	
	\textbf{Reasoning For Assignment}\\
The motivation behind this assignment was to introduce students to modules in Linux, familiarize students with the RAM disk device driver, and acquaint students with the Linux Crypto API. 	
		
	\textbf{Design and Programming Approach}\\
For the design of the RAM Disk Device without cryptography we followed the steps outlined by Pat Patterson. 
He provided a working copy of a RAM Disk Device Block Driver which we used for our project. 
For the cryptographic implementation, we referred to the documentation provided by kernel.org on the Linux Crypto API. 
Initially, we attempted to base our cryptographic implementation based on the tcrypt file found in our linux yocto directory. 
However, the API used in this file was from the Synchronous Block Cipher which has been deprecated. 
Further research led us to the Single Block Cipher API and the kernel.org documentation which we used to implement cryptography.

	\textbf{Testing Methodology}\\
		\textbf{RAM Disk Device Driver - Without Crypto}
 		After creating the appropriate file and editing the appropriate Makefile and Kconfig for our new module, we validated its correctness by ensuring the virtual machine would accept the new module. 
 		We did this by using the command insmod filename to load the module into our virtual machine. 
 		We then attempted to partition the new RAM Disk and mount it. 
 		Finally, we created a new file on the RAM Disk before unmounting and removing the module.
 		
 		\textbf{RAM Disk Device Driver - With Crypto}
 		For the most part we followed the same methodology as above with a few slight differences. 
 		The new Ram Disk Device Driver contained multiple print kernel statements that allowed us to view block data before encryption, after encryption, and after decryption. 
 		We compared the unencrypted data to the decrypted data to determine the validity of our module. 
 		If the unencrypted data did not match the decrypted data we would know there was a bug in our program. 
 		\\ 
	\textbf{Lessons Learned}\\
This homework assignment taught us about modules, RAM disk device drivers, and how to make use of the Cryptographic API in Linux. 
Perhaps most importantly, we learned how effective it is to plan ahead on projects for which we lack familiarity. 
Apart from some trouble with incorporating the newly created module into Linux, we were able to follow the steps we outlined in our first meeting with little issue. 
By breaking the project up into sub-problems we were able to easily identify what we needed to get done, how to go about completing the sub-problem, and compare our current progress to our intended progress.
		\\

	\clearpage

	\section{Version Control Log}
	
\begin{center}

\begin{tabular}{| m{2cm} | m{8cm} | m{4cm} | } 
\hline
 Date & Commit & Message \\ [0.5ex] 
 \hline\hline
  2016-11-12 & a49ccdc & Fix bug. Encrypt kinda weird. \\
 \hline
  2016-11-11 & 736a912 & Add encryption - buggy \\
 \hline
  2016-11-09 & 49370f4 & Add tex template \\
 \hline
   2016-11-02 & 599d4b5 & Add rdd.c skeleton code \\
 \hline
 
\end{tabular}

\end{center}
\hfill\\
N.B. We are using: git log --pretty=format:"\%h \%ad\%x09\%an\%x09\%s" --date=short
	
\clearpage

\section{Work Log}
\begin{center}
\begin{tabular}{ |m{2cm}|m{10cm}|m{2cm}| }
\hline
When & What & Duration \\ \hline
11/3/2016
& Met at KEC to discuss the project and identify what needed to be done to complete the project. 
We also researched information about Linux RAM disk device drivers and started working on the code for the assignment.
& 3 hours\\ 
\hline

11/7/2016 
& Using several resources we discovered online we created the first version of our module. 
We spent additional time attempting to implement the module into the virtual machine. 
& 3 hours\\ 
\hline

11/9/2016 
& Based on advice from Professor McGrath we were finally able to incorporate the module into the virtual machine and run our tests.
& 4 hours\\ 
\hline

11/11/2016 
& Researched the Linux Cryptographic API and based on its documentation we were able to implement an updated version of our module, this time with cryptography, into our virtual machine. 
& 3 hours\\ 
\hline

11/12/2016 
& Tested the cryptographic API and worked on fixing a bug with the decryption portion of our module.
& 3 hours\\
\hline

\end{tabular}
\end{center}
\end{document}

