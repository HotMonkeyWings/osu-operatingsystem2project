\documentclass[letterpaper,10pt,draftclsnofoot,titlepage,onecolumn]{IEEEtran}

\usepackage{graphicx}                                        
\usepackage{amssymb}                                         
\usepackage{amsmath}                                         
\usepackage{amsthm}                                          

\usepackage{alltt}                                           
\usepackage{float}
\usepackage{color}
\usepackage{url}

\usepackage{balance}
\usepackage[TABBOTCAP, tight]{subfigure}
\usepackage{enumitem}
\usepackage{pstricks, pst-node}
\usepackage{array}

\usepackage{geometry}
\geometry{textheight=8.5in, textwidth=6in}


\newcommand{\cred}[1]{{\color{red}#1}}
\newcommand{\cblue}[1]{{\color{blue}#1}}

\newcommand{\toc}{\tableofcontents}


\def\name{Group 29}

\parindent = 0.0 in
\parskip = 0.1 in

\begin{document}
	
	\begin{titlepage}
		\centering
		{\scshape\LARGE \textbf{CSS 444 HW 2 - I/O Elevators}\par}
		{\LARGE\itshape \textbf{Kyle Collins and Jonathan Harijanto}\par}
		{\LARGE Fall 2016\par}
		{\LARGE \today\par}
		
		{\LARGE Abstract\par}
		The purpose of this paper is to describe the work in implementing a CLOOK scheduler for Linux and to describe what we learned from the process. This paper will cover the reasoning for the assignment, our general design plan, testing methodology, and finally lessons learned. 
		
	\end{titlepage}
	
	\clearpage
	\tableofcontents
	
	\clearpage
	\section{SSTF Design Plan}
	
	Our general design plan will primarily focus on adding to the queue. If the queue is properly sorted by the time it gets to dispatch then the dispatch function should work properly. Our general algorithm is as follows
	
	If queue is empty, insert new request into queue.\\
	Else\\
		Get read/write head position\\
		Iterate through queue\\
		Compare new request to read/write head position.\\
		If new request l.t. r/w head pos\\
			if new request l.t. iterator position\\
				 Insert new request after iterator position\\
		Else if new request l.t.e. r/w head position\\
			if new request l.t.e. iterator position\\
				Insert new request before iterator position\\

	\clearpage
	\section{Questions}
	
	\textbf{Reasoning For Assignment}\\
	The purpose of this assignment is to reinforce the different scheduling algorithms, particularly that of LOOK and CLOOK, and to familiarize the student with how to interact with an unknown code base. The algorithm for CLOOK and LOOK aren't particularly difficult, instead its the technical detail of how to implement these algorithms within Linux that is intended to be the main challenge for the student.\\
		
	\textbf{Design and Programming Approach}\\
		Our initial approach involved deciding whether to implement CLOOK or LOOK. We decided on implementing CLOOK as that would be easiest. CLOOK is a scheduling algorithm that executes tasks only when it is traveling in one direction, as opposed to look which can execute tasks
		in both directions. The tasks that CLOOK will execute are based on the current position of the read/write head and the intended direction of travel. 
		
		The next phase of the assignment was for us to understand the functionality that comes with the linux scheduler. We reviewed documentation concerning lists, particularly list\_sort.h and list.h to understand how they worked. We also reviewed the NOOP scheduler and the elevator file.
		
		Our general algorithm for CLOOK was to sort the queue in a descending order. This would help with the implementation of CLOOK as it only travels in one direction and by dispatching to tasks in a descending order the scheduler would be traveling in one direction. We took advantage of the list\_sort function to do so. \\

	\textbf{Testing Methodology}\\
		Our testing methodology involved testing the scheduler for both one task and multiple tasks. For one task it was relatively simple. The scheduler would simply dispatch to the one task that was added. For the multiple tasks, the results of the dispatch were examined to see if they were in descending order, which they were. \\ 

	\textbf{Lessons Learned}\\
	The most challenging aspect of this assignment was understanding the existing code base and utilizing it to implement the scheduler. We learned a great deal on how to interact with code, some of which is poorly documented. The assignment did reinforce the concepts behind the scheduling algorithms but for the most part our primary learning took place when reading documentation and reviewing the code base.\\


	\section{Version Control Log}
	
\begin{center}

 \begin{tabular}{| m{2cm} | m{5cm} | m{2cm} | m{2cm} | m{3cm} | } 
\hline
 Date & Commit & Insertions & Deletions & Message \\ [0.5ex] 
 \hline\hline
  10/5/16 & aababe0da9b6a2f88d 4d771448854ca6c41f12ef & 18282136 & 0 & yocto! \\
 \hline
  10/5/16 & 063effe1f90dd2e7b9 7e8b8bdf5145f20ff44cfe & 3 & 0 & Create README.md \\
 \hline
  10/5/16 & ea4da8698f3836538d ef8cdf7673d19a7a700e3e & 41 & 0 & Add homework1 c \\
 \hline
  10/5/16 & afa6edf92fef775084 4b858e4b98f7f42c96a841 & 210 & 8 & Add mt.h and modify homework1.c \\
 \hline
  10/5/16 & d0b44c238e2c4285b0 4dba1d1c63d23e5a719553 & 16 & 0 & Start doing Producer() \\
 \hline
  10/6/16 & 496310f187be10b620 098e8682c8d49075c6394b & 56 & 4 & Group work late night session \\
 \hline
   10/7/16 & 21f928d426d4c052c e6301ffdc8a0768783ed4ee & 19 & 17 & Almost done, still buggy \\
 \hline
   10/9/16 & 3cb1a3713a54a1a10 af60650826931208acf1e10 & 16 & 15 & Add pthread destroy because I forgot to do it \\
 \hline
   10/9/16 & e72d7993e808dad667 8b0227e425f3cc05d6158d & 185 & 0 & Add files via upload - Clean up comments and display\\
 \hline
   10/9/16 & d04d2b85ef304db10f 1502169b498d07ea0544f9 & 258 & 0 & Add files via upload - latex files rough draft\\
 \hline
   10/9/16 & d2def80fe95399cba7 f72696a7ec78f59faf63dd & 54 & 30 & Fix!\\
 \hline
 
\end{tabular}
\end{center}
	
\clearpage

\section{Work Log}
\begin{center}
\begin{tabular}{ |m{2cm}|m{10cm}|m{2cm}| }
\hline
When & What & Duration \\ \hline
10/18/2016 
& We met at Kelly Computer Lab to discuss about the content of the assignments and the requirements to be done. We decided to implement "Pair Programming" again since both of us weren't familiar with the elevator concept at that time. We spent two hours to understand the different type of schedulers. 
& 2 hours\\ 
\hline

10/20/2016 
& We had a better understanding about the homework after we listened to some questions asked by classmates. Thus, after the class, we met a Kelly to create an algorithm for SSTF in a piece of paper. Furthermore, we also had to do some research about the noop-iosched code because it wasn't easy to read and understand every structs, pointers and variables in that code.  
& 3 hours\\ 
\hline

10/21/2016 
& We started to test our SSTF algorithm by modifying the copy of noop-iosched.c code. First, we changed every function name that has the word noop into sstf. Next, we modified the noop struct because sstf is going to need more variables.  In our design concept, we thought that the only function that needed major modification was the sstf\_dispatch. Thus we spent a lot of time trying to convert our algorithm into code carefully. We used the rest of our time to fix compile errors and do research to disable vitio.
& 5 hours\\ 
\hline

10/22/2016 
& We encountered a difficulty in enabling our scheduler into the virtual machine. We either got a kernel panic or received an error message when switching the scheduler.
Since this is not a common problem, we were having a difficulty finding helpful information from the internet. Thus, we had to spend most of our time only to fix this issue. Being able to switch the scheduler is important to us because we wanted to know whether our SSTF code works or not. 
& 5 hours\\ 
\hline

10/23/2016 
& We tried to fix the virtual machine issue and make a modification to any logic error in our SSTF code. After several hours of doing it, we felt that the current code has met the assignment requirements. Thus, we continued by creating a patch file and writing the LaTex documentation.
& 5 hours\\
\hline
\end{tabular}
\end{center}
\end{document}
