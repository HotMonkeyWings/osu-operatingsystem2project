\documentclass[letterpaper,10pt,draftclsnofoot,titlepage,onecolumn]{IEEEtran}

\usepackage{graphicx}                                        
\usepackage{amssymb}                                         
\usepackage{amsmath}                                         
\usepackage{amsthm}                                          

\usepackage{alltt}                                           
\usepackage{float}
\usepackage{color}
\usepackage{url}

\usepackage{balance}
\usepackage[TABBOTCAP, tight]{subfigure}
\usepackage{enumitem}
\usepackage{pstricks, pst-node}
\usepackage{array}

\usepackage{geometry}
\geometry{textheight=8.5in, textwidth=6in}


\newcommand{\cred}[1]{{\color{red}#1}}
\newcommand{\cblue}[1]{{\color{blue}#1}}

\newcommand{\toc}{\tableofcontents}


\def\name{Group 29}

\parindent = 0.0 in
\parskip = 0.1 in

\begin{document}
	
	\begin{titlepage}
		\centering
		{\scshape\LARGE \textbf{CSS 444 HW 2 - I/O Elevators}\par}
		{\LARGE\itshape \textbf{Kyle Collins and Jonathan Harijanto}\par}
		{\LARGE Fall 2016\par}
		{\LARGE \today\par}
		
		{\LARGE Abstract\par}
		WRITE SOMETHING HERE!
		
	\end{titlepage}
	
	\clearpage
	\tableofcontents
	
	\clearpage
	\section{SSTF Design Plan}
	
	WRITE SOMETHING HERE!

	\clearpage
	\section{Questions}
	
	\textbf{Reasoning For Assignment}\\
	The reasoning for this assignment is to become familiar with I/O schedulers. We were expected to understand these schedulers both higher (definition \& algorithm) and lower level (c code). We were also required to understand the scheduler implementation by asking to go deep into the kernel and switch the scheduler. Now we understand how the scheduler sorts and merges a given task from our testing.	\\
		
	\textbf{Design and Programming Approach}\\
		WRITE SOMETHING
	\\

	\textbf{Testing Methodology}\\
		WRITE SOMETHING
	\\ 

	\textbf{Lessons Learned}\\
	This assignment is hard and vague. We had to 'google' most of the instruction mentioned in the assignment including the virtio. However, at the same time, we gained a wealth of information about I/O schedulers, kernel patches, and Qemu build. After we finished coding the sstf scheduler, now we had a better understanding about the scheduler works especially LOOK. \\

	\clearpage

	\section{Version Control Log}
	
\begin{center}

 \begin{tabular}{| m{2cm} | m{5cm} | m{2cm} | m{2cm} | m{3cm} | } 
\hline
 Date & Commit & Insertions & Deletions & Message \\ [0.5ex] 
 \hline\hline
  10/5/16 & aababe0da9b6a2f88d 4d771448854ca6c41f12ef & 18282136 & 0 & yocto! \\
 \hline
  10/5/16 & 063effe1f90dd2e7b9 7e8b8bdf5145f20ff44cfe & 3 & 0 & Create README.md \\
 \hline
  10/5/16 & ea4da8698f3836538d ef8cdf7673d19a7a700e3e & 41 & 0 & Add homework1 c \\
 \hline
  10/5/16 & afa6edf92fef775084 4b858e4b98f7f42c96a841 & 210 & 8 & Add mt.h and modify homework1.c \\
 \hline
  10/5/16 & d0b44c238e2c4285b0 4dba1d1c63d23e5a719553 & 16 & 0 & Start doing Producer() \\
 \hline
  10/6/16 & 496310f187be10b620 098e8682c8d49075c6394b & 56 & 4 & Group work late night session \\
 \hline
   10/7/16 & 21f928d426d4c052c e6301ffdc8a0768783ed4ee & 19 & 17 & Almost done, still buggy \\
 \hline
   10/9/16 & 3cb1a3713a54a1a10 af60650826931208acf1e10 & 16 & 15 & Add pthread destroy because I forgot to do it \\
 \hline
   10/9/16 & e72d7993e808dad667 8b0227e425f3cc05d6158d & 185 & 0 & Add files via upload - Clean up comments and display\\
 \hline
   10/9/16 & d04d2b85ef304db10f 1502169b498d07ea0544f9 & 258 & 0 & Add files via upload - latex files rough draft\\
 \hline
   10/9/16 & d2def80fe95399cba7 f72696a7ec78f59faf63dd & 54 & 30 & Fix!\\
 \hline
 
\end{tabular}
\end{center}
	
\clearpage

\section{Work Log}
\begin{center}
\begin{tabular}{ |m{2cm}|m{10cm}|m{2cm}| }
\hline
When & What & Duration \\ \hline
10/18/2016 
& We met at Kelly Computer Lab to discuss about the content of the assignments and the requirements to be done. We decided to implement "Pair Programming" again since both of us weren't familiar with the elevator concept at that time. We spent two hours to understand the different type of schedulers. 
& 2 hours\\ 
\hline

10/20/2016 
& We had a better understanding about the homework after we listened to some questions asked by classmates. Thus, after the class, we met a Kelly to create an algorithm for SSTF in a piece of paper. Furthermore, we also had to do some research about the noop-iosched code because it wasn't easy to read and understand every structs, pointers and variables in that code.  
& 3 hours\\ 
\hline

10/21/2016 
& We started to test our SSTF algorithm by modifying the copy of noop-iosched.c code. First, we changed every function name that has the word noop into sstf. Next, we modified the noop struct because sstf is going to need more variables.  In our design concept, we thought that the only function that needed major modification was the sstf\_dispatch. Thus we spent a lot of time trying to convert our algorithm into code carefully. We used the rest of our time to fix compile errors and do research to disable vitio.
& 5 hours\\ 
\hline

10/22/2016 
& We encountered a difficulty in enabling our scheduler into the virtual machine. We either got a kernel panic or received an error message when switching the scheduler.
Since this is not a common problem, we were having a difficulty finding helpful information from the internet. Thus, we had to spend most of our time only to fix this issue. Being able to switch the scheduler is important to us because we wanted to know whether our SSTF code works or not. 
& 5 hours\\ 
\hline

10/23/2016 
& We tried to fix the virtual machine issue and make a modification to any logic error in our SSTF code. After several hours of doing it, we felt that the current code has met the assignment requirements. Thus, we continued by creating a patch file and writing the LaTex documentation.
& 5 hours\\
\hline
\end{tabular}
\end{center}
\end{document}

