\documentclass[letterpaper,10pt,draftclsnofoot,titlepage,onecolumn]{IEEEtran}

\usepackage{graphicx}                                        
\usepackage{amssymb}                                         
\usepackage{amsmath}                                         
\usepackage{amsthm}                                          

\usepackage{alltt}                                           
\usepackage{float}
\usepackage{color}
\usepackage{url}

\usepackage{balance}
\usepackage[TABBOTCAP, tight]{subfigure}
\usepackage{enumitem}
\usepackage{pstricks, pst-node}
\usepackage{array}

\usepackage{geometry}
\geometry{textheight=8.5in, textwidth=6in}


\newcommand{\cred}[1]{{\color{red}#1}}
\newcommand{\cblue}[1]{{\color{blue}#1}}

\newcommand{\toc}{\tableofcontents}


\def\name{Group 29}

%% The following metadata will show up in the PDF properties
%\hypersetup{
%   colorlinks = false,
%   urlcolor = black,
%   pdfauthor = {\name},
%   pdfkeywords = {cs444 ``operating systems 2'' Homework 1},
%   pdftitle = {CS 444 Project 1: Getting Acquainted},
%   pdfsubject = {CS 444 Project 1},
%   pdfpagemode = UseNone
%}

\parindent = 0.0 in
\parskip = 0.1 in

\begin{document}
	
	\begin{titlepage}
		\centering
		{\scshape\LARGE \textbf{CSS 444 HW 1 - Concurrency}\par}
		{\LARGE\itshape \textbf{Kyle Collins and Jonathan Harijanto}\par}
		{\LARGE Fall 2016\par}
		{\LARGE \today\par}
		
		{\LARGE Abstract\par}
		The purpose of this paper is to describe the work involved in creating the Yocto Kernel and Producer-Consumer program. The Yocto Kernel is to be used for the
		CS444 course taught by Professor Kevin McGrath at Oregon State University. The Producer-Consumer program was written in C and took advantage of POSIX threading.
		This paper covers the commands used to create the Yocto Kernel as well as a brief description of relevant QEMU flags. Finally, this paper includes a section on
		how the Producer-Consumer program was approached and completed. 
		
	\end{titlepage}
	
	\clearpage
	\tableofcontents
	\clearpage
	\section{Command Log}
		\newcounter{rowcount}
		\setcounter{rowcount}{0}
		\begin{tabular}{@{\stepcounter{rowcount}\therowcount.)\hspace*{\tabcolsep}}ll}	
			mkdir CS444-029 \\
			
			cd CS444-029 \\
			
			git clone git://git.yoctoproject.org/linux-yocto-3.14 \\
			
			cd linux-yocto-3.14	\\		
			
			git checkout v3.14.26 \\	
			
			source /scratch/opt/environment-setup-i586-poky-linux.csh \\	
					
			cp /scratch/fall2016/files/config-3.14.26-yocto-qemu .config \\
			
			cp /scratch/fall2016/files/bzImage-qemux86.bin /scratch/fall2016/CS444-029/linux-yocto-3.14/ \\
			
			cp /scratch/fall2016/files/core-image-lsb-qemux86.ext3 /scratch/fall2016/CS444-029/linux-yocto-3.14/ \\
			
			make -j4 all \\
			
			qemu-system-i386 -gdb tcp::5529 -S -nographic -kernel "arch/x86/boot/bzImage" \newline -drive file=core-image-lsb-sdk-qemux86.ext3,if=virtio -enable-kvm -net none -usb -localtime --no-reboot --append "root=/dev/vda rw console=ttyS0 debug" \\
			
			In different terminal - Open 'gdb' \\
			
			(gdb) Target remote:5529 \\
			
			(gdb) continue \\
			
		\end{tabular}

	\clearpage
	\section{Concurrency Write-up}
	
	\textbf{Reasoning For Assignment}\
		The reasoning for this assignment is twofold. The first reason is to set up the virtual machine for later use in the class. Many of the future assignments in this class will make use of the VM and it is important to familiarize ourselves with it early so that we do not negatively impact the work of other groups and so that we can focus on the core lessons of those future assignments. The second reason for this assignment is to serve as an introduction to concurrency. Concurrency is an important concept in Computer Science as it allows for greater efficiency in certain tasks.\\
		
	\textbf{Design and Programming Approach}\
		Our initial approach to the problem involved learning about multithreading. Using the class materials as a guide, specifically the pthread\_hello.c file,
		we were able to view relevant documentation concerning POSIX threads. Once we created a simple producer-consumer program we moved on to tackling the assignment.
		We modified our original program to include randomly generated sleep times for both the producer and consumer. In order to do this, we used the Mersenne Twister
		random number generator. This was chosen over the rdrand random number generator because we will have to demo this program on os-class and os-class does not support
		rdrand. The Mersenne Twister RNG was simple enough to understand and use. After the implementation of Mersenne Twister and the sleep period we encountered a scheduling
		error. Through debugging we were able to discover that occasionaly the producer would wake up the consumer when the producer woke up from sleep. This meant that even when the
		consumer was supposed to be sleeping as part of the "consumption" phase, if the producer woke up during the middle of this phase it would wake the consumer as well. We solved
		this problem by moving our wait condition in the consumer to after the "consumption" phase. We also modified some of the logic in our program to better determine empty buffers
		or a full buffer. By focusing on these areas we were able to resolve our scheduling errors. Finally, we had our last bug with the consumer displaying the wrong
		value if the producer created a new item while the consumer slept. Essentially, the consumer was displaying the value of the item at the top of the stack
		and not the value of the item it had actually consumed. To fix this, we created a temp structure to hold the appropiate values for the consumer. So when
		the consumer began its "consumption" period it saved the appropiate values and displayed those instead of the values at the top of the stack. \\

	\textbf{Testing Methodology} 
		Our testing methodology was a combination of a automatic testing and manual testing. We also tested edge cases for an empty buffer and full buffer.
		For manual testing we used hard coded sleep values for both the producer and consumer. We had cases where the producer would sleep longer than the consumer,
		in which case the index would never increase past 0 because the consumer would always be ready to grab the next item. We had cases where the consumer would
		sleep longer than the producer which would guarantee that the buffer would fill and the producer should become blocked. The result of these test cases showed
		the bug described above where the consumer was printing the value at the top of the stack as opposed to the correct value. It is because of these manual tests
		we were able to discover this bug. Automatic testing involved running the program for a long time and making note of the values that were consumed and produced.
		In theory, if the program ran for long enough the buffer should fill at least once in which case the producer should be blocked. Using automatic testing we
		were able to discover the bug involving the consumer waking up to soon.\\ 

	\textbf{Lessons Learned}
		We learned a great deal from this homework assignment. Aside from the POSIX commands and the Mersenne Twister commands needed to use this assignment we
		learned how important scheduling was. A great deal of our initial bugs stemmed from scheduling errors which took quite some time to fix. This programming
		assignment reinforced the importance of state-diagrams which had we used would have made this assignment much easier to complete.\\ 

	
	\clearpage
	\section{Qemu Flags}
	\textbf{-gdp tcp::5529} : Instruct the system to wait for GDB connection on TCP with a port number of 5,529. \\
	
	\textbf{-S} : Instruct CPU to not start during startup. \\
	
	\textbf{-nographic} : Instruct QEMU to run in command line form. \\
	
	\textbf{-enable-KVM} : Enables KVM (Kernel Virtual Machine) for Linux. \\
	
	\textbf{-net none} : Enable the USB driver. \\
	
	\textbf{-localtime} : Use system local time. \\
	
	\textbf{--no-reboot} : Exit instead of rebooting. \\
	
	\textbf{--append "root=/dev/vda rw console=ttyS0 debug"} : Allows for extra options in the kernel command line. \\
	
	\clearpage
	\section{Version Control Log}
	
\begin{center}
 \begin{tabular}{| m{2cm} | m{5cm} | m{2cm} | m{2cm} | m{3cm} | } 
 \hline
 Date & Commit & Insertions & Deletions & Message \\ [0.5ex] 
 \hline\hline
  10/5/16 & aababe0da9b6a2f88d 4d771448854ca6c41f12ef & 18282136 & 0 & yocto! \\
 \hline
  10/5/16 & 063effe1f90dd2e7b9 7e8b8bdf5145f20ff44cfe & 3 & 0 & Create README.md \\
 \hline
  10/5/16 & ea4da8698f3836538d ef8cdf7673d19a7a700e3e & 41 & 0 & Add homework1 c \\
 \hline
  10/5/16 & afa6edf92fef775084 4b858e4b98f7f42c96a841 & 210 & 8 & Add mt.h and modify homework1.c \\
 \hline
  10/5/16 & d0b44c238e2c4285b0 4dba1d1c63d23e5a719553 & 16 & 0 & Start doing Producer() \\
 \hline
  10/6/16 & 496310f187be10b620 098e8682c8d49075c6394b & 56 & 4 & Group work late night session \\
 \hline
   10/7/16 & 21f928d426d4c052c e6301ffdc8a0768783ed4ee & 19 & 17 & Almost done, still buggy \\
 \hline
   10/9/16 & 3cb1a3713a54a1a10 af60650826931208acf1e10 & 16 & 15 & Add pthread destroy because I forgot to do it \\
 \hline
   10/9/16 & e72d7993e808dad667 8b0227e425f3cc05d6158d & 185 & 0 & Add files via upload - Clean up comments and display\\
 \hline
   10/9/16 & d04d2b85ef304db10f 1502169b498d07ea0544f9 & 258 & 0 & Add files via upload - latex files rough draft\\
 \hline
   10/9/16 & d2def80fe95399cba7 f72696a7ec78f59faf63dd & 54 & 30 & Fix!\\
 \hline
 
\end{tabular}
\end{center}
	
\clearpage

\section{Work Log}
\begin{center}
\begin{tabular}{ |m{2cm}|m{10cm}|m{2cm}| }
\hline
When & What & Duration \\ \hline
10/4/2016 & We met for an hour at Kelly Computer Lab to discuss about the content of the assignments, the requirements to be done, and the method of programming that we are going to use. In the end, we decided that "Pair Programming" is the most efficient way for this first assignment. Furthermore, we also managed to get the overall concepts of  and kernel build. & 1 hours\\ \hline
10/5/2016 & We had our second meeting at Kelly Computer Lab again. This time, we directly cloned the yocto repo and followed all the needed steps to run the kernel. We also set up github private repo to contain these OS2 FILES. It took us around 2 hours before we could move on with the concurrency assignment. We were able to start this concurrency assignment because we started from reading Kevin's sample code pthread\_hello, and from there we started to google about phtreads. & 3 hours\\ \hline
10/6/2016 & We started to code producer and consumer functions because we thought that our concept was correct. With the help of different sources like: linux.die.org, pubs.opengroup.org, and stackoverflow, these two functions were done (well, sort of). & 4 hours\\ \hline
10/7/2016 & We wanted to finish all the functions today because we're just missing main and signal handler function. However, we didn't expect that actually there were a lot of bugs when we ran the code as a whole. Thus, we ended up debugging all the errors for the rest of the meeting. & 3 hours\\ \hline
10/8/2016 & We realized that the code is still buggy, but we remembered that documentation is also part of the assignment. Thus, we decided to learn LaTex by creating this document together. Arguably, today is a self-study LaTex session. & 2 hours\\ \hline
10/9/2016 & We tried our best to fix and clean up our code so it is executable. After two hours of doing it, we felt that our current code is following Kevin's requirement already. Thus, we spend the rest of the time to finish up the LaTex documentation. & 4 hours\\
\hline
\end{tabular}
\end{center}
\end{document}

