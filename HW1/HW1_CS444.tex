\documentclass[letterpaper,10pt,draftclsnofoot,titlepage,onecolumn]{IEEEtran}

\usepackage{graphicx}                                        
\usepackage{amssymb}                                         
\usepackage{amsmath}                                         
\usepackage{amsthm}                                          

\usepackage{alltt}                                           
\usepackage{float}
\usepackage{color}
\usepackage{url}

\usepackage{balance}
\usepackage[TABBOTCAP, tight]{subfigure}
\usepackage{enumitem}
\usepackage{pstricks, pst-node}
\usepackage{array}

\usepackage{geometry}
\geometry{textheight=8.5in, textwidth=6in}

%random comment

\newcommand{\cred}[1]{{\color{red}#1}}
\newcommand{\cblue}[1]{{\color{blue}#1}}

\newcommand{\toc}{\tableofcontents}

%\usepackage{hyperref}

\def\name{D. Kevin McGrath}

%pull in the necessary preamble matter for pygments output
%\input{pygments.tex}

%% The following metadata will show up in the PDF properties
%\hypersetup{
%   colorlinks = false,
%   urlcolor = black,
%   pdfauthor = {\name},
%   pdfkeywords = {cs444 ``operating systems 2'' Homework 1},
%   pdftitle = {CS 444 Project 1: Getting Acquainted},
%   pdfsubject = {CS 444 Project 1},
%   pdfpagemode = UseNone
%}

\parindent = 0.0 in
\parskip = 0.1 in

\begin{document}
	
	\begin{titlepage}
		\centering
		{\scshape\LARGE \textbf{CSS 444 HW 1 - Concurrency}\par}
		{\LARGE\itshape \textbf{Kyle Collins and Jonathan Harijanto}\par}
		{\LARGE Fall 2016\par}
		{\LARGE \today\par}
		
		{\LARGE Abstract\par}
		Write stuff here when done with hw. 
		
	\end{titlepage}
	
	\clearpage
	\tableofcontents
	
	%input the pygmentized output of mt19937ar.c, using a (hopefully) unique name
	%this file only exists at compile time. Feel free to change that.
	
	\clearpage
	\section{Command Log}
		\newcounter{rowcount}
		\setcounter{rowcount}{0}
		\begin{tabular}{@{\stepcounter{rowcount}\therowcount.)\hspace*{\tabcolsep}}ll}	
			mkdir CS444-029 \\
			
			cd CS444-029 \\
			
			git clone git://git.yoctoproject.org/linux-yocto-3.14 \\
			
			cd linux-yocto-3.14	\\		
			
			git checkout v3.14.26 \\	
			
			source /scratch/opt/environment-setup-i586-poky-linux.csh \\	
					
			cp /scratch/fall2016/files/config-3.14.26-yocto-qemu .config \\
			
			cp /scratch/fall2016/files/bzImage-qemux86.bin /scratch/fall2016/CS444-029/linux-yocto-3.14/ \\
			
			cp /scratch/fall2016/files/core-image-lsb-qemux86.ext3 /scratch/fall2016/CS444-029/linux-yocto-3.14/ \\
			
			make -j4 all \\
			
			qemu-system-i386 -gdb tcp::5529 -S -nographic -kernel "arch/x86/boot/bzImage" \newline -drive file=core-image-lsb-sdk-qemux86.ext3,if=virtio -enable-kvm -net none -usb -localtime --no-reboot --append "root=/dev/vda rw console=ttyS0 debug" \\
			
			In different terminal - Open 'gdb' \\
			
			(gdb) Target remote:5529 \\
			
			(gdb) continue \\
			
		\end{tabular}

	\clearpage
	\section{Concurrency Write-up}
	
	\textbf{Reasoning For Assignment}\
		The reasoning for this assignment is twofold. The first reason is to set up the virtual machine for later use in the class. Many of the future assignments in this class will make use of the VM and it is important to familiarize ourselves with it early so that we do not negatively impact the work of other groups and so that we can focus on the core lessons of those future assignments. The second reason for this assignment is to serve as an introduction to concurrency. Concurrency is an important concept in Computer Science as it allows for greater efficiency in certain tasks.\\
		
	 

	
	\clearpage
	\section{Qemu Flags}
	\textbf{-gdp tcp::5529} : Instruct the system to wait for GDB connection on TCP with a port number of 5,529. \\
	
	\textbf{-S} : Instruct CPU to not start during startup. \\
	
	\textbf{-nographic} : Instruct QEMU to run in command line form. \\
	
	\textbf{-enable-KVM} : Enables KVM (Kernel Virtual Machine) for Linux. \\
	
	\textbf{-net none} : Enable the USB driver. \\
	
	\textbf{-localtime} : Use system local time. \\
	
	\textbf{--no-reboot} : Exit instead of rebooting. \\
	
	\textbf{--append "root=/dev/vda rw console=ttyS0 debug"} : Allows for extra options in the kernel command line. \\
	
	\clearpage
	\section{Version Control Log}
	
\begin{center}
\begin{tabular}{ |m{2cm}|m{5cm}|m{1cm}|m{1cm}|m{5cm}| }
 \hline
 Date & Commit & Insertions & Deletions & Message \\ [0.5ex] 
 \hline\hline
  10/5/16 & aababe0da9b6a2f88d4d771448854ca6c41f12ef & 18282136 & 0 & yocto! \\
 \hline
  10/5/16 & 063effe1f90dd2e7b97e8b8bdf5145f20ff44cfe & 3 & 0 & Create README.md \\
 \hline
  10/5/16 & ea4da8698f3836538def8cdf7673d19a7a700e3e & 41 & 0 & Add homework1 c \\
 \hline
  10/5/16 & afa6edf92fef7750844b858e4b98f7f42c96a841 & 210 & 8 & Add mt.h and modify homework1.c \\
 \hline
  10/5/16 & d0b44c238e2c4285b04dba1d1c63d23e5a719553 & 16 & 0 & Start doing Producer() \\
 \hline
  10/6/16 & 496310f187be10b620098e8682c8d49075c6394b & 56 & 4 & Group work late night session \\
 \hline
   10/7/16 & 21f928d426d4c052ce6301ffdc8a0768783ed4ee & 19 & 17 & Almost done, still buggy \\
 \hline
   10/9/16 & 3cb1a3713a54a1a10af60650826931208acf1e10 & 16 & 15 & Add pthread destroy because I forgot to do it \\
 \hline
   10/9/16 & e72d7993e808dad6678b0227e425f3cc05d6158d & 185 & 0 & Add files via upload - Clean up comments and display\\
 \hline
   10/9/16 & d04d2b85ef304db10f1502169b498d07ea0544f9 & 258 & 0 & Add files via upload - latex files rough draft\\
 \hline
   10/9/16 & d2def80fe95399cba7f72696a7ec78f59faf63dd & 54 & 30 & Fix!\\
 \hline
 
\end{tabular}
\end{center}
	
\clearpage

\section{Work Log}
\begin{center}
\begin{tabular}{ |m{2cm}|m{10cm}|m{2cm}| }
\hline
When & What & Duration \\ \hline
10/4/2016 & We met for an hour at Kelly Computer Lab to discuss about the content of the assignments, the requirements to be done, and the method of programming that we are going to use. In the end, we decided that "Pair Programming" is the most efficient way for this first assignment. Furthermore, we also managed to get the overall concepts of  and kernel build. & 1 hours\\ \hline
10/5/2016 & We had our second meeting at Kelly Computer Lab again. This time, we directly cloned the yocto repo and followed all the needed steps to run the kernel. We also set up github private repo to contain these OS2 FILES. It took us around 2 hours before we could move on with the concurrency assignment. We were able to start this concurrency assignment because we started from reading Kevin's sample code pthread\_hello, and from there we started to google about phtreads. & 3 hours\\ \hline
10/6/2016 & We started to code producer and consumer functions because we thought that our concept was correct. With the help of different sources like: linux.die.org, pubs.opengroup.org, and stackoverflow, these two functions were done (well, sort of). & 4 hours\\ \hline
10/7/2016 & We wanted to finish all the functions today because we're just missing main and signal handler function. However, we didn't expect that actually there were a lot of bugs when we ran the code as a whole. Thus, we ended up debugging all the errors for the rest of the meeting. & 3 hours\\ \hline
10/8/2016 & We realized that the code is still buggy, but we remembered that documentation is also part of the assignment. Thus, we decided to learn LaTex by creating this document together. Arguably, today is a self-study LaTex session. & 2 hours\\ \hline
10/9/2016 & We tried our best to fix and clean up our code so it is executable. After two hours of doing it, we felt that our current code is following Kevin's requirement already. Thus, we spend the rest of the time to finish up the LaTex documentation. & 4 hours\\
\hline
\end{tabular}
\end{center}
	

	
%	\subsection{blah}
%	\subsubsection{yada yada}
%	This is a paragraph in \LaTeX.
	
%	This is a new paragraph.
	
%	\begin{itemize}
%		\item \begin{equation}
%			\label{eq1}
%			\int_0^\pi \sin(x) \partial x
%		\end{equation}
%		\item $\backslash$ As seen in Eq. \ref{eq1}, % blah blah blah
%	\end{itemize}

	
%	\emph{\textbf{\color{red}This is italicized and red}}
%	\tableofcontents
	
%	\section*{Appendix 1: Source Code}
%	\input{__mt19937ar.c.tex}
	
\end{document}
